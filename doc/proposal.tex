\documentclass{article}

\usepackage{pxfonts}

\author{Alex Vall\'ee \and Matan Nassau}
\title{libyaml++:  A Proposal for a COMP 446 Project}

\begin{document}

\maketitle

\section{Introduction}
For our COMP 446 project we chose to write a C++ library for reading and writing \emph{YAML} streams.  YAML stands for \emph{YAML Ain't a Markup Language}.  It is a portable data-serialization protocol, designed primarily for readability by humans.  YAML plays well as a protocol for message-based distributed processing systems.

YAML is somewhat comparable with XML in that it can be used as a protocol for serializing structured, heirarchical documents.  XML, however, was designed to be backwards-compatible with \emph{SGML}, the \emph{Standard Generalized Markup Language}, which in turn was designed primarily to store structured documents.  YAML therefore has more freedom of design.  When compared to XML, YAML elevates its design goals to support data streams that show a balance between a flexible syntax and readability.

\end{document}
