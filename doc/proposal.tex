\documentclass{article}

\usepackage{pxfonts}
\usepackage{url}
\usepackage{float}

\author{Alex Vall\'ee \and Matan Nassau}
\title{libyaml++:  A Proposal for a COMP 446 Project}

\begin{document}

\maketitle

\section{Introduction}
For our COMP 446 project we chose to write a C++ library for reading and writing \emph{YAML} streams.  YAML stands for \emph{YAML Ain't a Markup Language}.  It is a portable data-serialization protocol, designed primarily for readability by humans.  YAML plays well as a protocol for message-based distributed processing systems.

YAML is somewhat comparable with XML in that it can be used as a protocol for serializing structured, heirarchical documents.  XML, however, was designed to be backwards-compatible with \emph{SGML}, the \emph{Standard Generalized Markup Language}, which in turn was designed primarily to store structured documents.  YAML therefore has more freedom of design.  When compared to XML, YAML elevates its design goals to support data streams that show a balance between a flexible syntax and readability.

\floatstyle{ruled}
\newfloat{Figure}{tbp}{yamlsample}[section]
\begin{Figure}
  \begin{verbatim}
  ---
  - name:  Richard Stallman
    email: rms@gnu.org
  - name:  Linus Torvalds
    email: torvalds@linux-foundation.org
  - name:  Jeff Atwood
    email: jatwood@codinghorror.com
  \end{verbatim}
  \caption{Sample YAML data}
\end{Figure}

\section{Goals}

We consider two primary goals in this project:

\begin{itemize}
  \item Learning
  \item Writing a useful and quality YAML library
\end{itemize}

\subsection{Learning Goals}
We intend on learning the STL and templates programming. We also intend on familiaralizing ourselves with the upcoming \emph{C++0x} standard by experimenting with \emph{TR1}.  We plan on emphasizing on test-driven development by writing thorough unit tests before starting any new features.  Additionally, we will use an automated build tool and a \emph{distributed} version control system to practice project build and code management.

\subsection{Software Requirements}
The goal of this project is to write a useful, quality library for handling YAML streams.  We lay our priorities in decreasing order as follows:

\begin{enumerate}
  \item \textbf{Correct}.  We consider our code correct if each implemented feature works as described by the latest YAML standard.  Indeed, this suggests we will make sure each added feature is correct before we move on to the next feature.  We will utilize aggressive testing for this purpose.
  \item \textbf{Useful}.  We consider our library useful if it can read a simple YAML document into an internal representation, and write an internal representation into a simple YAML document.  We consider a simple YAML document to hold strings, sequences and mappings.
  \item \textbf{Feature-rich}.  Finally, given a set of useful features implemented correctly, we will approach a completely standard-compliant library.  Indeed, this is a very ambitious goal, and with the standard being constantly evolving --- even more so.  Nevertheless, this goal will set the iterative and constant-change nature of our project, giving us always something to look forward to.
\end{enumerate}

\section{C++ Use}

In this project, we expect to use a broad array of C++ techniques and features appropriate for library writing:

\begin{itemize}
  \item Templates and meta-programming
  \item Inheritance and polymorphism
  \item Various design patterns, such as factory method, singleton and so on.
  \item Massive use of the STL containers and algorithms
  \item Some TR1 elements, such as \emph{tr1::shared\_ptr}
  \item Unit-testing framework:  \emph{Google Test}
\end{itemize}

We will also use \emph{waf} for our build system, and \emph{git} for our version control system.

\section{Deliverables}

The following is a rough outline of our planned iterations:

\begin{enumerate}
  \item Begin/End document marks
  \item Comments
  \item Simple Scalars
  \item Sequences
  \item Simple Mappings
  \item Tagged nodes
  \item Untagged nodes (type resolution on runtime)
  \item Verbatim, Folded Scalars
  \item Complex Mappings
  \item Unicode support
\end{enumerate}

The above list probably includes more iterations that time permits;  nevertheless, these encompass a reasonable portion of YAML standard and lay a rough roadmap for an implementation.
\end{document}
