\documentclass{article}

\usepackage{setspace}
\doublespacing

\usepackage{pxfonts}
\usepackage[usenames,dvipsnames]{color}
\usepackage[colorlinks=true,linkcolor=black,filecolor=black,citecolor=black,urlcolor=black]{hyperref}

\author{Alex Vall\'ee \and Matan Nassau}
\title{COMP 446 Team Project Report:  libyaml++}

\begin{document}
\maketitle

\tableofcontents
\pagebreak

\section{Overview}
libyaml++ is a C++ library for reading and writing YAML streams.  YAML stands
for \emph{YAML Ain't a Markup Language}.  It is a portable data-serialization
protocol, designed primarily for readability by humans.  YAML plays well as a
protocol for message-based distributed processing systems, as well as a
language for configuration files, log files and so on.

\section{Motivation}
For this project we wanted to write something different than what we have
written in classes before:  a library.  We also wanted this library to involve
parsing or some aspect of compiler design.  YAML came up as an idea while
discussing web development with Rails.  We ran a fast lookup and
originally\footnote{We later found
\href{http://code.google.com/p/yaml-cpp/}{yaml-cpp} } found no C++ library for
YAML;  thus we made a decision.

\section{Issues Encountered}
During our work on the project we encountered a few problems and considered
various design alternatives.

\subsection{Code Problems}
Code problems are those that lie low in the code design of the library.  These 
are solved often by design patterns, miscellanous coding techniques or choices 
of third-party tools and libraries.

\subsubsection{Parsing}
Our experience in parsing was limited at the beginning of the term.  We have
previously written a small compiler in Perl using mostly regular expressions for
the lexical analysis.  Because writing a lexical analyzer in C++ is quite
a different task from writing a lexical analyzer in Perl using regular
expressions there was not much experience we could draw from.  We have also
gained significant amount of theoretical knowledge from COMP 335, but bringing
theory to practice still is a challenge.  Thus, we had to do a lot of readings
on topics related to compiler design in general and C++ implementations in
particular.  The \emph{Dragon Book}\footnote{\emph{Compilers :  Principles,
Techniques, \& Tools}, 2e, by Alfred V. Aho, Monica S. Lam, Ravi Sethi and
Jeffrey D. Ullman.} in particular was of great use for us.  We have also looked
at other YAML implementations as listed on the \href{http://yaml.org/}{YAML
website}.

The end result is a system composed of 4 components:

\begin{description}
  \item[The Scanner] represents the lexical analyzer, translating a stream of 
    characters into a stream of tokens.
  \item[The Parser] represents the syntactic analyzer, translating a stream of 
    tokens into an internal representation of the YAML data.
  \item[The Representation] encodes the YAML data in an internal structure 
    iterable through the library API.
  \item[The Native Representation] makes the internal representation more
    intuitive for a user to use.
\end{description}

To implement the scanner we considered two approaches.  We discuss them in 
\S{} \ref{scan_function}.  To implement the parser we used a top-down recursive 
descent approach.

\subsubsection{Representation}
One fundamental issue we had was specifically with the fourth component advised 
by the YAML standard, called the Native component.  The YAML standard advises 
on serving the user with a further translation of the internal representation 
into a data structure that is native to the programming language.  The user 
will thus work with the familiar primitives and data structures of the 
programming language he or she is accustomed to.  This works very well with 
dynamically typed languages such as Ruby and Python.  In C++, however, we must
specify a single type for all elements of a container, and the common approach
to hold various types in a single container is through polymorphism.  Following
this approach we ended up with the same data as in the internal representation,
only wrapped in a native data structure rather than a custom internal object
structure.

The choice between the two data structures is thus more subtle than in dynamic 
languages.  Even more so is the case if we provide an iteration mechanism for 
the internal representation that feels native to the STL;  in this case, 
although the representation is not native it does feel so for the user.  To 
implement the internal representation we considered two alternatives.  The 
first is node-based hierarchy, and the second is an event-driven handler.

\begin{description}
  \item[Node-Based Hierarchy] involves a composite object structure that allows 
    storing the YAML data in an abstract iterable tree structure.  This implies 
    the user must learn a few API types:  a \emph{node} base class, a 
    \emph{sequence} composite type, a \emph{string} leaf type and so on.  But 
    it also means that the user can flexibly iterate over the entire structure, 
    and perhaps even manipulate it.
  \item[Event-Driven Handler] presents a simpler mechanism in which the user 
    need only be familiar with a single base \emph{handler} class.  The parser 
    notifies the user through calls to this base class about events that took 
    place while parsing.  For example, if an integer is encountered during a 
    parse, the parser emits the \texttt{handler::on\_integer(int)} event.  The 
    user can listen for this event by subclassing the handler class, overriding 
    the \texttt{on\_integer(int)} function, and passing an object of the new 
    type to the parser.
\end{description}

The choice between these two approaches is mostly a tradeoff between 
flexibility and simplicity.  On the one hand a node-based hierarchy delivers a 
fine-grained flexibility to the user.  The user can iterate over a pre-parsed 
YAML data structure, while remaining aware of scope and context.  On the other 
hand the event-handling mechanism offers a simple interface consisting of a 
single base class, rendering the API much simpler and more elegant.  The 
node-based hierarchy also supports multi-pass traversal:  the user may traverse 
repeatedly the same structure, whereas in the case of the event-driven 
mechanism receiving the events again implies parsing again.

\subsubsection{The \texttt{lex::scanner::scan()} Function}
\label{scan_function}
The classic way to write a lexical analyzer is through a single scan() 
function, branching into the various states through a switch statement.  The 
advantage of following this approach is that it is widely acceptable and 
idiomatic, it is very well documented and is thus easy to implement correctly.  
The disadvantage of this approach that might show if the grammar is complex 
enough is that the function tends to grow fast and get harder to understand and
maintain.  As undergraduate students we are vastly humbled by the giants before
us who have proven this approach correct, and thus we implemented it as
described in the texts.  Nevertheless, we branched off in our revision control
system and took the adventurous path of implementing the scanning procedure in
an alternative manner.

By breaking the different rules each token presents into classes and objects, 
we ended up with a much simpler scan() function.  What we implemented is a 
variation of the Chain of Responsibility pattern of the GoF.  The scan() 
function holds a list of tokens, where each token points to the next.  The list 
is ordered in the same order the cases are ordered in the original switch 
statement of the classic approach.  The token classes all derive from a common 
ancestor, which declares a virtual \texttt{scan()} function and holds a reference to the 
input stream.  Each token knows how to scan itself out of a stream.  The 
scanner therefore asks the first token in the list to scan itself;  if it can't 
it restores the stream to its original state and forwards the request to the
next scanner in the list.

% \subsubsection{portability:  stl (boost is less standard but de-facto standard nonetheless!)}

\subsubsection{The Face of the API}
In the spirit of making things as simple as possible, we were looking for ideas 
on simplifying the API of our library as much as possible.  This agrees with 
the STL philosophy of minimizing the number of entities (functions, classes) by 
abstracting as much as possible.  We settled on revealing two free functions to 
the user:  \texttt{yaml::load()} and \texttt{yaml::dump()}.  The first loads a 
stream of characters and translates it into an internal representation, while 
the latter dumps an internal representation into a stream of YAML characters.

\subsection{Peripheral Problems}

% \subsubsection{switch from waf to boost.build}

\subsubsection{working on multiple machines without administrative access:
one-command build}
Because our project utilizes a number of external tools for the development 
process, a couple of additional steps are required in order to set up a 
development environment for working on libyaml++.  The two tools are 
Boost.Build and GoogleTest.  The former is required for easily building the 
library and for easing other build tasks.  The latter is required for running 
unit tests.  We should stress that these are not required for developing 
\emph{with} the library, but for for developing \emph{the library itself}.  
Along the development of our project we were working on multiple machines:  
school labs, school servers, home, etc.  In most cases we had no administrative 
privileges on the machines.  We thus felt the need to automate the process of 
obtaining the tools and building the library.

The result of this is the \textit{build.sh} script, a Bash script suitable for 
Unix-flavoured workstations.  The script downloads the development tools and 
installs them in a directory most systems offer for this sort of tasks --- 
typically the temporary directory.  The script also makes sense to be used as 
the build script for end-users, as it does make building libyaml++ much easier.

% \subsubsection{revision control}

% \subsubsection{code quality: testing}

\section{Instructions}

\subsection{Quickstart}

libyaml++ is a C++ library for reading and writing YAML streams.  To build,
run from your Linux command shell:

\begin{verbatim}
~$ tar xzf libyamlpp.tar.gz
~$ cd libyamlpp
libyamlpp$ ./build.sh
\end{verbatim}

Next, see \S{} \ref{using_libyamlpp} for sample code.

\subsection{Directory Tree Structure}

This distribution consists of three components.  The first and most important
component is the library's source code, and it is placed under the src/
subdirectory.  The second component is the demo application, which demonstrates
the library's capabilities and use and is located, obviously enough, in the
demo/ subdirectory.  The last component is the testing code, which tests the
source code of the library and is located in the test/ subdirectory.

\subsection{Building}

The easiest way to build libyaml++ is using the \textit{build.sh} script in the
distribution's root directory.  It will set up your system for building
libyaml++ as well as its unit-tests.  The \textit{build.sh} script assumes you
are using Linux though.  An alternative way to get the library built is to
build the library manually.  If you intend to hack on the code we recommend
you set yourself up properly: install Boost.BuildV2 and GoogleTest in a
convenient directory in your system and use \emph{bjam} for the build.

We will cover in this section all approaches mentioned above.

\subsubsection{Building Using \textit{build.sh}}

From the command prompt of your Linux box run:

\begin{verbatim}
~$ cd libyamlpp
libyamlpp$ ./build.sh
\end{verbatim}

The \textit{build.sh} script can take any target as a command-line parameter, so you may
for example run

\begin{verbatim}
libyamlpp$ ./build.sh demo check
\end{verbatim}

to build the demo and check targets, for the demonstration application and
unit-tests, repectively.  The \textit{build.sh} script can also take a special
command-line argument, wipeclean, which will wipe clean any files and
directories it may have created:

\begin{verbatim}
libyamlpp$ ./build.sh wipeclean
\end{verbatim}

The above command will leave nothing but the original libyaml++ distribution
has.  Make sure you pass no other command-line argument when running the above
command.

\subsubsection{Using Boost.Build}

Boost.Build doesn't usually come by default in an operating-system's install.
So first, make sure you have Boost.BuildV2, and Boost.Jam on which Boost.Build
depends, installed.  Once that is set up, running the build is a matter of a
single command.


\paragraph{Setting-Up Boost.Build}

If you run a Debian-based Linux distribution, this may simply be a matter of

\begin{verbatim}
#$ sudo apt-get install boost-build
\end{verbatim}

Other Linux distributions have their own package management system, so read the
appropriate manual pages.  If that fails or if you don't have a package
management system, get Boost.Build from
\url{http://www.boost.org/doc/tools/build/index.html} and follow the instructions
at \url{http://www.boost.org/doc/tools/build/doc/html/bbv2/installation.html}.
Installing Boost.Build is a fairly straightforward procedure that consists
mainly of four steps:

\begin{enumerate}
  \item Copy the extracted Boost.Build distribution into a directory reachable by
    your operating-system's \texttt{PATH}.
  \item Configure Boost.Build by choosing your compiler of choice in the
    site-config.jam file in Boost.Build's root directory.  It should be a
    matter of uncommenting a line such as

   \begin{verbatim}using gcc ;\end{verbatim}

  \item Copy the \emph{bjam} binary into a directory reachable by your
    operating-system's \texttt{PATH}.
  \item \texttt{export BOOST\_BUILD\_PATH=/path/to/boost-build}
\end{enumerate}

Further details are given in the link above.

\paragraph{Building Using Boost.Build}

To build libyaml++, from the libyaml++ distribution's root directory run the
command:

\begin{verbatim}
libyamlpp$ bjam release
\end{verbatim}

This will build libyaml++ and place the resultant binary in a directory
structure under src/bin/release/.  The exact hierarchy structure will depend on
your system.  Our system, for example, will place the resultant binary in
src/bin/release/gcc/libyaml++.so.  Your environment may be, for example, in
src/bin/release/msvc/libyaml++.dll.

To build the demo application, run the following command from the project
distribution's root directory:

\begin{verbatim}
libyamlpp$ bjam demo release
\end{verbatim}

Again, the exact location of the resultant binary will depend on your system
but will be somewhere under demo/bin/release/.  In our case it is in
demo/bin/release/gcc/libyaml++-demo.

To build the test suites, run

\begin{verbatim}
libyamlpp$ bjam check
\end{verbatim}

This will also run the tests once they are built.

\subsection{Testing}

The source code comes with a set of tests that attempt to find bugs or issues
with the library.  These tests rely on the Google Test framework, so you must
have that set up to run them.

\subsubsection{Setting-up GoogleTest}

Get GoogleTest from \url{http://code.google.com/p/googletest/}.  We used GoogleTest
v1.1.0 successfully.  Extract the distribution package in a directory of your
choice, and build it.  To install GoogleTest in your home directory on a Unix
machine the following commands will do:

\begin{verbatim}
~$ tar jxf gtest-1.1.0.tar.bz2
~$ cd gtest-1.1.0
gtest-1.1.0$ ./configure --prefix=$HOME/usr
gtest-1.1.0$ make
gtest-1.1.0$ make install
\end{verbatim}

Next, you must make sure that

\begin{enumerate}
\item the C++ preprocessor can find the gtest headers
\item the linker can find the gtest library when linking the libyaml++ tests
\item the runtime evironment can find the shared library when the tests are run
\end{enumerate}

To accomodate for this on a Unix platform using gcc for the build and assuming
the above commands were used for installing GoogleTest, add the following three
environment variables to your shell's startup file, such as \$HOME/.bashrc:

\begin{verbatim}
export CPLUS_INCLUDE_PATH=$HOME/usr/include:$CPLUS_INCLUDE_PATH
export LIBRARY_PATH=$HOME/usr/lib:$LIBRARY_PATH
export LD_LIBRARY_PATH=$HOME/usr/lib:$LD_LIBRARY_PATH
\end{verbatim}

\subsubsection{Running The Tests}

Once GoogleTest is set up you may run the tests.  From the commandline and from
within the project's root directory, run:

\begin{verbatim}
libyamlpp$ bjam check
\end{verbatim}

\subsection{Using libyaml++}
\label{using_libyamlpp}

To use libyaml++, you need a C++ compiler and \texttt{tr1::shared\_ptr}.  Put
the binaries somewhere reachable by your compiler and linker's search paths.

\begin{figure}
  \caption{Sample code using libyaml++}
  \begin{verbatim}
#include <iyaml++.hh>
#include <tr1/memory>
#include <iostream>
#include <stdexcept>

using namespace std;

class my_handler : public yaml::event_handler {
    void on_string(const string& s);
    void on_sequence_begin();
};

void my_handler::on_string(const string& s) {
    cout << "parsed  " << s << endl;
}

void my_handler::on_sequence_begin() {
    cout << "sequence about to begin." << endl;
}

int main(int argc, char* argv[])
{
    tr1::shared_ptr<my_handler> handler(new my_handler());
    while( cin ) {
        try { yaml::load(cin, handler); }

        // catch syntax error
        catch( const runtime_error& e ) {
            cerr << e.what() << endl;
        }
    }
    return 0;
}
\end{verbatim}
\end{figure}

\end{document}
